\documentclass[cn,10pt,math=newtx,citestyle=gb7714-2015,bibstyle=gb7714-2015]{elegantbook}

\usepackage{float}

\title{历年 CSP 题目解析}
\subtitle{仅为参考练习所用}

\author{lonlyn}
\institute{Shanxi University Algorithm Group}
\date{December 30, 2021}
\version{1.0}

\setcounter{tocdepth}{3}

\cover{cover.jpg}

% 本文档命令
\usepackage{array}
\newcommand{\ccr}[1]{\makecell{{\color{#1}\rule{1cm}{1cm}}}}

\definecolor{customcolor}{RGB}{32,178,170}
\colorlet{coverlinecolor}{customcolor}

\begin{document}

\maketitle
\frontmatter

\chapter*{特别声明}

\markboth{Introduction}{前言}

该书仅供内部学习使用,如果有侵权请联系作者。

信息学竞赛的发展,吸引越来越多的人加入了“卷”的行列。CCF CSP的历年题解在网上也是随处可见,但题解质量参差不齐。很多题解只有标准答案,缺少题目分析;更有甚者无法通过答案,充满了分号大小写问题等错误。

本书的目的是为了实现以下几点:

\begin{itemize}
  \item 提供规范的代码程序。这里的规范,既要具有程序的可读性,也要具备考场的简易性。
  \item 提供多样的解题思路。有些时候,网上的大佬往往一语道破问题求解的思路,但怎么想到的却往往不提。这里力求从部分分开始,逐渐深入,汇集众人智慧,逐步解决难题。
  \item 提供筛选的额外补充。做一道题的目的不是只做一道题,而是可以做到举一反三,但我们常常忽略这一点。
\end{itemize}

感谢 \href{https://github.com/ElegantLaTeX/ElegantBook}{Elegant\LaTeX{}} 提供如此精美的模板,希望这本书能够给大家带来帮助。

\vskip 1.5cm

\begin{flushright}
lonlyn\\
December 30, 2021
\end{flushright}

\tableofcontents

\mainmatter

\chapter{CCF CSP认证总览}

待补充。

\chapter{第23次认证(2021年9月)}

待补充。

\chapter{第24次认证(2021年12月)}

\section{题目及涉及知识点}

\begin{table}[htbp]
  \centering
  \begin{tabular}{ccc}
    \toprule
    题目编号 & 题目名称 & 知识点\\
    \midrule
    1 & 序列查询 & 数学\\
    2 & 序列查询新解 & 数学\\
    3 & 登机牌条码 & 模拟,多项式除法\\
    4 & 磁盘文件操作 & 线段树\\
    5 & 极差路径 & 树分治\\
    \bottomrule
  \end{tabular}
\end{table}

\section{202112-1 序列查询}

\subsection*{题目背景}

西西艾弗岛的购物中心里店铺林立,商品琳琅满目。为了帮助游客根据自己的预算快速选择心仪的商品,IT 部门决定研发一套商品检索系统,支持对任意给定的预算 $x$,查询在该预算范围内($\le x$)价格最高的商品。如果没有商品符合该预算要求,便向游客推荐可以免费领取的西西艾弗岛定制纪念品。

假设购物中心里有 $n$ 件商品,价格从低到高依次为 $A_1,A_2,\cdots, A_n$,
,则根据预算 $x$ 检索商品的过程可以抽象为如下序列查询问题。

\subsection*{题目描述}

$A=[A_0,A_1,A_2,\cdots,A_n]$ 是一个由 $n+1$ 个 $[0,N)$ 范围内整数组成的序列,满足 $0=A_0<A_1<A_2<\cdots<A_n<N$
。(这个定义中蕴含了 $n$ 一定小于 $N$。)

基于序列 $A$,对于 $[0,N)$ 范围内任意的整数 $x$,查询 $f(x)$ 定义为:序列 $A$ 中{\heiti 小于等于} $x$ 的整数里最大的数的{\heiti 下标}。具体来说有以下两种情况:

\begin{enumerate}
  \item 存在下标 $0\le i<n$ 满足 $A_i\le x < A_{i+1}$,此时序列 $A$ 中从 $A_0$ 到 $A_i$ 均小于等于 $x$,其中最大的数为 $A_i$,其下标为 $i$,故 $f(x)=i$。
  \item $A_n\le x$,此时序列 $A$ 中左右的数都小于等于 $x$,其中最大的数是 $A_n$,故 $f(x)=n$。
\end{enumerate}

令 $sum(A)$ 表示 $f(0)$ 到 $f(N-1)$ 的总和,即:

\begin{equation*}
sum(A) = \sum_{i=0}^{N-1} {f(i)=f(0)+f(1)+f(2)+\cdots +f(N-1)}
\end{equation*}

对于给定的序列 $A$,试计算 $sum(A)$。

\subsection*{输入格式}

从标准输入读入数据。

输入的第一行包含空格分隔的两个正整数 $n$ 和 $N$。

输入的第二行包含 $n$ 个用空格分隔的整数 $A_1,A_2,\cdots,A_n$
。

注意 $A_0$
 固定为 $0$,因此输入数据中不包括 $A_0$ 
。

\subsection*{输出格式}

输出到标准输出。

仅输出一个整数,表示 $sum(A)$ 的值。

\subsection*{样例}

输入\#1:

\begin{lstlisting}
3 10
2 5 8
\end{lstlisting}

输出\#1:

\begin{lstlisting}
15
\end{lstlisting}

解释\#1:

$A=[0, 2, 5, 8]$

\begin{table}[H]
  \centering
  \begin{tabular}{ccccccccccc}
    \toprule
    $i$ & $0$ & $1$ & $2$ & $3$ & $4$ & $5$ & $6$ & $7$ & $8$ & $9$ \\
    $f(i)$ & $0$ & $0$ & $1$ & $1$ & $1$ & $2$ & $2$ & $2$ & $3$ & $3$ \\
    \bottomrule
  \end{tabular}
\end{table}

如上表所示,$sum(A)=f(0)+f(1)+\cdots + f(9)=15$。

考虑到 $f(0)=f(1)$、$f(2)=f(3)=f(4)$、$f(5)=f(6)=f(7)$ 以及 $f(8)=f(9)$,亦可通过如下算式计算 $sum(A)$;

\begin{equation*}
  sum(A)=f(0)\times 2+f(2)\times 3+f(5)\times 3 + f(8)\times 2
\end{equation*}

输入\#2:

\begin{lstlisting}
9 10
1 2 3 4 5 6 7 8 9
\end{lstlisting}

输出\#2:

\begin{lstlisting}
45
\end{lstlisting}

\subsection*{子任务}

$50$ \% 的测试数据满足 $1\le n\le 200$ 且 $n\le N\le 1000$;

全部的测试数据满足 $1\le n\le 200$ 且 $n\le N\le 10^7$。

\subsection*{提示}

若存在区间 $[i,j)$ 满足 $f(i)=f(i+1)=\cdots=f(j-1)$,使用乘法运算 $f(i)\times (j-i)$ 代替将 $f(i)$ 到 $f(j-1)$ 逐个相加,或可大幅提高算法效率。


\subsection{$50$\% 数据——模拟}

\subsubsection{思路}

模拟一下这个过程,计算出每一个 $f(i)$ 后加起来即可。

考虑针对确定的 $x$,如何求解 $f(x)$。我们可以从小到大枚举 $A$ 中的数,枚举到第一个大于等于 $x$ 的数即可。注意末尾的判断。

枚举 $x$ 时间复杂度 $\mathbf{O}(N)$,计算 $f(x)$ 时间复杂度 $\mathbf{O}(n)$,整体时间复杂度 $\mathbf{O}(nN)$。

\subsubsection{C++实现}

待补充。

\subsection{$100$\% 数据——利用 $f(x)$ 单调性}

\subsubsection{思路}

为了方便,设 $f(n+1) = \infty$。

通过模拟,可以得到一个显然的结论:

\begin{theorem}[$f(x)$的单调性] \label{thm:fx_monotonicity} 
  对于 $x,y\in [0,N)$,若 $x \le y$,则 $f(x) \le f(y)$。
\end{theorem}

那么,我们可以从小到大枚举 $x$,同时记录目前 $f(x)$ 的值,设为 $y$,那么 $A_{y+1}$ 是第一个大于 $x$ 的数。
当需要计算 $f(x+1)$ 的时候,我们从小到大依次判断 $A_{y+1},A_{y+2},\cdots$ 是否满足条件,
直到遇到第一个大于 $f(x+1)$ 的数 $A_z$,那么 $f(x+1)=z-1$。
之后,在 $f(x+1)$ 的基础上以同样的步骤求 $f(x+2)$,直到求完所有的值。

考虑该算法的时间复杂度,枚举 $x$ 的复杂度是 $\mathbf{O}(N)$,
而 $A$ 数组中每个数对多被枚举一次,枚举所有 $x$ 的整体复杂度 $\mathbf{O}(n)$,
可以得到整体复杂度 $\mathbf{O}(N+n)$。

\subsubsection{C++实现}

\lstinputlisting[language=c++]{code/24/202112-1-100.cpp}

\subsection{$100$\% 数据——阶段求和}

\subsubsection{思路}

在提示中,指出了可以将 $f(x)$ 相同的值一起计算。现在需要解决的问题是如何快速确定 $f(x)$ 值相等的区间。

通过观察和模拟可以发现,随着 $x$ 增大,$f(x)$ 只会在等于某个 $A$ 数组的值时发生变化。
更具体的说,对于某个属于 $A$ 数组的值 $A_i$ 来说,$[A_i,A_{i+1}-1]$ 间的 $f(x)$ 值是相同的,
这样的数共有 $A_{i+1}-A_i$ 个。

也可以以另一种方式理解:对于一个值 $y$,考虑有多少 $x$ 满足 $f(x)=y$。
当 $x<A_y$ 时,$f(x)<y$,当 $x\ge A_{y+1}$ 时,$f(x)>y$。
只有 $x\in [A_y,A_{y+1}]$ 时才能得到 $f(x)=y$。

得到范围后,我们就可以根据 $A$ 数组来进行求和计算。

考虑 $f(x)=n$ 的处理:
我们可以得知满足 $f(x)=n$ 的 $x$ 共有 $N-A_n$ 个,
根据上文推算,我们可以将 $A_{n+1}$ 设置为 $A_n+(N-A_n)=N$ 即可等效替代。

时间复杂度 $\mathbf{O}(n)$。

\subsubsection{C++实现}

\lstinputlisting[language=c++]{code/24/202112-1-100-2.cpp}

\section{202112-2 序列查询新解}

\subsection*{题目背景}

上一题“序列查询”中说道:
$A=[A_0,A_1,A_2,\cdots,A_n]$ 是一个由 $n+1$ 个 $[0,N)$ 范围内整数组成的序列,满足 $0 = A_0 < A_1 < A_2 < \cdots < A_n < N$
。基于序列 $A$,对于 $[0,N)$ 范围内任意的整数 $x$,查询 $f(x)$ 定义为:序列 $A$ 中{\heiti{小于等于}} $x$ 的整数里最大的数的下标。

对于给定的序列 $A$ 和整数 $x$,查询 $f(x)$ 是一个很经典的问题,可以使用二分搜索在 $\mathbf{O}(\log n)$ 的时间复杂度内轻松解决。但在 IT 部门讨论如何实现这一功能时,小 P 同学提出了些新的想法。

\subsection*{题目描述}

小 P 同学认为,如果事先知道了序列 $A$ 中整数的分布情况,就能直接估计出其中小于等于 $x$ 的最大整数的大致位置。接着从这一估计位置开始线性查找,锁定 $f(x)$。如果估计得足够准确,线性查找的时间开销可能比二分查找算法更小。

比如说,如果 $A_1,A_2,\cdots,A_n$
 均匀分布在 $(0,N)$ 的区间,那么就可以估算出:
 
\begin{equation*}
    f(x)\approx \frac{(n+1)\cdot x}{N}
\end{equation*}

为了方便计算,小 P 首先定义了比例系数 $r=\lfloor \frac{N}{n+1} \rfloor$
 
,其中 $\lfloor \rfloor$ 表示下取整,即 $r$ 等于 $N$ 除以 $n+1$ 的商。进一步地,小 P 用 $g(x)=\lfloor \frac{x}{r}\rfloor$
 
 表示自己估算出的 $f(x)$ 的大小,这里同样使用了下取整来保证 $g(x)$ 是一个整数。

显然,对于任意的询问 $x\in [0,N)$,$g(x)$ 和 $f(x)$ 越接近则说明小 P 的估计越准确,后续进行线性查找的时间开销也越小。因此,小 P 用两者差的绝对值 $|g(x)-f(x)|$ 来表示处理询问 $x$ 时的误差。

为了整体评估小 P 同学提出的方法在序列 $A$ 上的表现,试计算:
 
\begin{equation*}
    error(A)=\sum\limits_{i=0}^{N-1}{|g(i)-f(i)|}=|g(0)-f(0)| + \cdots + |g(N-1)-f(N-1)|
\end{equation*}

\subsection*{输入格式}

从标准输入读入数据。

输入的第一行包含空格分隔的两个正整数 $n$ 和 $N$。

输入的第二行包含 $n$ 个用空格分隔的整数 $A_1,A_2,\cdots,A_n$
。

注意 $A_0$
 固定为 $0$,因此输入数据中不包括 $A_0$
。

\subsection*{输出格式}

输出到标准输出。

仅输出一个整数,表示 $error(A)$ 的值。

\subsection*{样例}

输入格式\#1:

\begin{lstlisting}
3 10
2 5 8
\end{lstlisting}

输出格式\#1:

\begin{lstlisting}
5
\end{lstlisting}

解释\#1:

$A=[0, 2, 5, 8]$

$r = \lfloor \frac{N}{n+1}\rfloor=\lfloor \frac{10}{3+1}\rfloor=2$

\begin{table}[H]
  \centering
  \begin{tabular}{ccccccccccc}
    \toprule
    % \midrule
    $i$ & $0$ & $1$ & $2$ & $3$ & $4$ & $5$ & $6$ & $7$ & $8$ & $9$ \\
    $f(i)$ & $0$ & $0$ & $1$ & $1$ & $1$ & $2$ & $2$ & $2$ & $3$ & $3$ \\
    $g(i)$ & $0$ & $0$ & $1$ & $1$ & $2$ & $2$ & $3$ & $3$ & $4$ & $4$ \\
    $|g(i)-f(i)|$ & $0$ & $0$ & $0$ & $0$ & $1$ & $0$ & $1$ & $1$ & $1$ & $1$ \\
    \bottomrule
  \end{tabular}
\end{table}

输入格式\#2:

\begin{lstlisting}
9 10
1 2 3 4 5 6 7 8 9
\end{lstlisting}

输出格式\#2:

\begin{lstlisting}
0
\end{lstlisting}

输入格式\#3:

\begin{lstlisting}
2 10
1 3
\end{lstlisting}

输出格式\#3:

\begin{lstlisting}
6
\end{lstlisting}

解释\#3:

$A=[0, 1, 3]$

$r = \lfloor \frac{N}{n+1}\rfloor=\lfloor \frac{10}{2+1}\rfloor=3$

\begin{table}[H]
  \centering
  \begin{tabular}{ccccccccccc}
    \toprule
    % \midrule
    $i$ & $0$ & $1$ & $2$ & $3$ & $4$ & $5$ & $6$ & $7$ & $8$ & $9$ \\
    $f(i)$ & $0$ & $1$ & $1$ & $2$ & $2$ & $2$ & $2$ & $2$ & $2$ & $2$ \\
    $g(i)$ & $0$ & $0$ & $0$ & $1$ & $1$ & $1$ & $2$ & $2$ & $2$ & $3$ \\
    $|g(i)-f(i)|$ & $0$ & $1$ & $1$ & $1$ & $1$ & $1$ & $0$ & $0$ & $0$ & $1$ \\
    \bottomrule
  \end{tabular}
\end{table}

\subsection*{子任务}

$70$ \% 的测试数据满足 $1\le n\le 200$ 且 $n\le N\le 1000$;

全部的测试数据满足 $1\le n\le 10^5$ 且 $n\le N\le 10^9$。

\subsection*{提示}

需要注意,输入数据 $[A_1\cdots A_n]$
 并不一定均匀分布在 $(0,N)$ 区间,因此总误差 $error(A)$ 可能很大。



\end{document}
